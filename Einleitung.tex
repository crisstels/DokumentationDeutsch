\section{Einleitung}
Die folgende Projektdokuentation schildert den Ablauf eines Beispielprojektes für den Deutschunterricht.
\subsection{Projektbeschreibung}
Im Herbst isst man gerne zu einer Tasse Kaffe oder heißen Kakao ein Stück Kuchen. Nun ist zu Hause nicht ständig Kuchen verfügbar. Daher soll im Laufe des Projekts
ein Kuchen entstehen.
\subsection{Projektziel}
Ziel dieses Projektes ist es einen leckeren Kuchen zu backen, welchen man im Herbst zu einer Tasse Kaffe oder heißen Kakao genießen kann.
Dabei soll der Kuchen aus Zutaten bestehen, die bereits zu Hause verfügbar sind. Dies gilt auch für die Auswahl der Backform.
Der Kuchen sollte aus saisonalen Früchten bestehen und lecker schmecken. Zudem sollte der Kuchen schnell und einfach zubereitet werden,
damit man diesen innerhalb eines Tages zubereiten und auch verzehren kann. Alle Anforderungen an das Projekt können im Lastenheft nachgelesen werden.
Ein Auszug aus dem Lastenheft befindet sich im Anhang A:Lastenheft.

\section{Projektplanung}
In der Projektplanung soll die benötigte Zeit und die notwendigen Ressource, sowie ein Ablauf der Durchführung 
des Projektes geplant werden.
\subsection{Projektphasen}
Für die Umsetzung des Projektes standen der Autorin sechs Stunden zur Verfügung. Diese wurden vor Projektbeginn in verschiedene Phasen
unterteilt, die während der Kuchen Zubereitung durchlaufen werden. Die Einteilung der verschiedenen Phasen, sowie eine grobe Zeitplanung lässt sich aus Table 1: Grobe Zeitplannung entnehmen.
\begin{table}[h]
    \centering
    \begin{tabular}{l*{2}{r}r|}
        \hline
        Projektphase    & Geplante Zeit \\
        \hline
        Planung & 3 h\\
        Durchführung    & 2 h  \\
        Test      & 1 h   \\
        \hline
        Gesamt & 6 h \\
        \hline
        \end{tabular}
        \caption{Grobe Zeitplanung}
\end{table}
\subsection{Ressourcenplanung}
Eine Übersicht der genutzen Ressourcen befindet sich im Anhang unter blabl. Dort
sind alle Ressourcen aufgelistet, die während dieses Projektes eingesetzt wurden. Darunter
zählen Material, Zutaten, sowie Personal. Bei den Zutaten wurde darauf geachtet, dass vorallem schon bereits gekaufte Lebensmittel verarbeitet werden.
Falls noch Lebensmittel zusätzlich benötigt wurden, wurde beim kauf darauf geachtet, dass vorzugsweise Bio-Lebensmittel gekauft wurden.
Dies hält zum einen die Projektkosten gering und zum anderen wird gleichzeitig auf aktuelle ökolgische Umstände geachtet.
\subsection{Projektkosten}
Die Projektkosten, die während der Zubereitung des Kuchens anfallen, werden hier kalkuliert.
Neben den Personalkosten, die durch die Realisierung des Projekts enstehen, müssen auch die Kosten für das Material und die Zutaten berücksichtigt werden.
Pauschal werden 5€ als Personalkosten pro Stunde definiert, da dieses Projekt ausschließlich von einer einzelnen Person durchgeführt wird.

Die Gesamtkosten für die Durchführung des Projektes, sowie die Kosten der einzelnen Vorgängen lassen sich aus Table2: Kostenaufstellung entnehmen.

\begin{table}[h]
    \centering
    \begin{tabular}{l*{5}{r}r|}
        \hline
        Vorgang    & Zeit  &    Ressourcen & Personal & Kosten\\
        \hline
        Zutaten besorgen & 2 h & 4€ & 5€ & 18€\\
        Kuchenteig herstellen und backen    & 1,5 h & 1€ & 5€  & 9€ \\
        Kuchen dekorieren      & 0,5 h & 1€ & 5€ & 3€   \\
        Abnahme & 0,5 h & 0€ & 5€ & 5€ \\
        \hline
        Gesamt & & & & 30€ \\
        \hline
        \end{tabular}
        \caption{Kostenaufstellung}
\end{table}
\section{Projektdurchführung}
\subsection{Ablauf}
Bevor mit der eigentlichen Projektdurchführung begonnen wurde, wurde zunächst ein geeignetes Rezept(footnote) aus dem Internet rausgesucht. Das Rezept Apfelkuchen in Kastenform konnte alle Anforderungen aus dem Lastenheft erfüllen und wurde daher ausgewählt.
Anschließend wurden die benötigten Hilfsmittel, wie Handrührgerät, Schüssel, Löffel, etc. und Zutaten rausgesucht. Dabei ist aufgefallen, dass einige Zutaten
noch gar nicht vorhanden waren. Also wurden diese fehlenden Zutaten ersteinmal besorgt. Nachdem alle Zutaten besorgt worden waren und alle nötigen Hilfsmittel betriebsbereit waren, konnte mit der
Zubereitung des Kuchens begonnen werden. Der Kuchenteig wurde nach den Angaben des Rezeptes(footnote) zubereitet. Im Anschluss wurde der Kuchenteig im Ofen gebacken.
Währenddessen wurde mittels der Stäbchenprobe(footnote) getestet, ob der Kuchen bereits verzehrfertig sei. Bei erfolgreicher Durchführung des eben genannten Testes, wurde der Kuchen aus dem Ofen rausgenommen und zum Abkühlen auf ein Blech gestellt.
Nachdem der Kuchen vollständig ausgekühlt war, konnte dieser mit Zuckerguss dekoriert werden. Am Ende erfolgte ein Geschmackstest.
\subsection{Probleme}
Während der Durchführung ist, aufgefallen, dass nicht alle Zutaten sofort verfügbar waren. Einige Dinge, wie beispielsweise Äpfel und Zucker mussten noch im Supermarkt gekauft werden. Daher konnte mit der Zubereitung später als ursprünglich geplant gestartet werden.
Zudem musste der Kuchen länger im Ofen bleiben, als im Rezept angegeben war. Zehn Minuten nach der angegebenen Backzeit, war die Stäbchenprobe erfolgreich.
\section{Evaluation}
\subsection{Soll-/Ist-Vergleich}
Bei einer rückblickenden Betrachtung des Projektes, kann festgehalten werden, dass alle zuvor festgelegten Anforderungen gemäß dem Pflichtenheft erfüllt wurden. Der zu Beginn des Projektes im Abschnitt 2.1 (Projektphasen) erstellte Zeitplan konnte insgesamt eingehalten werden.
Ein Soll-/Ist-Vergleich, welcher im Anhang unter (Soll-Vergleich) zu finden ist zeigt, dass es nur geringfügige Abweichungen vom Zeitplan gab. Dieser Vergleich lässt erkennen, dass man in der Planungsphase eine Stunde zu viel eingeplant hatte. Gleichzeitig wurde in der Durchführungsphase, wegen der bereits genannten Problemen tatsächlich eine Stunde mehr gebraucht.
Dadurch konnte die in der Planung entstandene Differenz ausgeglichen werden.
\subsection{Lessons Learned}
Im Zuge des Projektes konnte die Autorin wertvolle Erfahrungen bzgl. der Planung und Durchführung von Projekten sammeln.
Dabei wurde deutlich, wie wichtige eine genaue Planung eines Projektes für die erfolgreiche Umsetzung ist. Außerdem konnten neue Erkenntnise in Bezug auf die Zubereitung eines Kuchens erlangt werden.
Die Stäbchenprobe erwies sich als besonders nützlich, um die tatsächlich Endbackzeit herauszufinden. Abschlißend lässt sich also sagen, dass die Realisierung dieses Projektes eine große Bereicherung für die Autorin war.
\subsection{Ausblick}
Obwohl alle im Lastenheft aufgelisteten Anforderungen erfüllt werden konnten, können in Zukunft, wenn dieser Kuchen nicht bereits aufgegessen wurde, dennoch weitere Anforderungen definiert werden. Es wurde bereits angefragt, ob es nicht möglich wäre, auf der Zuckerglasur noch Zimt hinzuzufügen. 
Außerdem wurde beispielsweise gefragt, ob man den Kuchen zusätzlich mit Sahne servieren könnte. Diese weiteren Anforderungen können problemlos übernommen und umgesetzt werden.
