\section{Einleitung}
Die folgende Projektdokuentation schildert den Ablauf eines Beispielprojektes für den Deutschunterricht.
\subsection{Projektbeschreibung}
Im Herbst isst man gerne zu einer Tasse Kaffe oder heißen Kakao ein Stück Kuchen. Nun ist zu Hause nicht ständig Kuchen verfügbar. Daher soll im Laufe des Projekts
ein Kuchen entstehen.
\subsection{Projektziel}
Ziel dieses Projektes ist es einen leckeren Kuchen zu backen, welchen man im Herbst zu einer Tasse Kaffe oder heißen Kakao genießen kann.
Dabei soll der Kuchen aus Zutaten bestehen, die bereits zu Hause verfügbar sind. Dies gilt auch für die Auswahl der Backform.
Der Kuchen sollte aus saisonalen Früchten bestehen und lecker schmecken. Zudem sollte der Kuchen schnell und einfach zubereitet werden,
damit man diesen innerhalb eines Tages zubereiten und auch verzehren kann.

\section{Projektplanung}
In der Projektplanung soll die benötigte Zeit und die notwendigen Ressource, sowie ein Ablauf der Durchführung 
des Projektes geplant werden.
\subsection{Projektphasen}
Für die Umsetzung des Projektes standen der Autorin sechs Stunden zur Verfügung. Diese wurden vor Projektbeginn in verschiedene Phasen
unterteilt, die während der Kuchen Zubereitung durchlaufen werden. Die Einteilung der verschiedenen Phasen, sowie eine grobe Zeitplanung lässt sich aus Tabelle 1: Grobe Zeitplannung entnehmen.
\begin{center}
    \begin{tabular}{|l*{2}{r}r|}
        \hline
        Projektphase    & Geplante Zeit \\
        \hline
        Planung & 3 h\\
        Durchführung    & 2 h  \\
        Test      & 1 h   \\
        \hline
        Gesamt & 6 h \\
        \hline
        \end{tabular}
\end{center}
\subsection{Ressourcenplanung}
Eine Übersicht der genutzen Ressourcen befindet sich im Anhang unter blabl. Dort
sind alle Ressourcen aufgelistet, die während dieses Projektes eingesetzt wurden. Darunter
zählen Material, Zutaten, sowie Personal. Bei den Zutaten wurde darauf geachtet, dass vorallem schon bereits gekaufte Lebensmittel verarbeitet werden.
Falls noch Lebensmittel zusätzlich benötigt wurden, wurde beim kauf darauf geachtet, dass vorzugsweise Bio-Lebensmittel gekauft wurden.
Dies hält zum einen die Projektkosten gering und zum anderen wird gleichzeitig auf aktuelle ökolgische Umstände geachtet.
\subsection{Projektkosten}
\subsubsection{Amortisationsdauer}
\subsection{Lastenheft}

\section{Projektdurchführung}
\subsection{Ablauf}
\subsection{Probleme}

\section{Evaluation}
\subsection{Soll-/Ist-Vergleich}
\subsection{Lessons Learned}
\subsection{Ausblick}